\definecolor{Text}{RGB}{0.36, 0.54, 0.66}
\definecolor{Element}{RGB}{0.53, 0.66, 0.42}
\definecolor{Attr}{RGB}{0.6, 0.4, 0.8}

\usepackage{listings}
\lstset{language=XML,
showspaces=false,
showtabs=false,
basicstyle=\ttfamily,
columns=fullflexible,
breaklines=true,
showstringspaces=false,
breakatwhitespace=true,
escapeinside={(*@}{@*)},
basicstyle=\ttfamily
}
\lstdefinelanguage{XML}
{
  morestring=[b]",
  morestring=[s]{>}{<},
  morecomment=[s]{<?}{?>},
  stringstyle=\color{Text},
  identifierstyle=\color{Element}\upshape,
  keywordstyle=\color{Attr}\bfseries,
  morekeywords={xmlns,version,type}% list your attributes here
}


\begin{lstinputlisting}% Start your code-block
<TEI>
<text>
<div canonicalRef="Phaedr.-:1,5" type="text">
<head>Phaedr. 1,5</head>
<div type="originaltext">
<head>
Vacca et
<hi type="Vokabelangabe">
capella
<note type="Vokabelangabe">capella,-ae f.: Ziege</note>
</hi>
, ouis et leo.
</head>
<div type="Abschnitt/Sinneinheit">
<head type="Gliederung">Promythion (1-2)</head>
<lg xml:lang="la">
<l met="u-|u-|u//-|u/-|u-|ux" n="1" real="#senar">
Numquam est
<hi rend="Vokabelangabe">fidelis<note type="Vokabelangabe"><seg xml:lang="la">fidelis,-e</seg>: verlässlich (siehe auch: <seg xml:lang="la">fides</seg>)</note></hi>
cum potente <hi rend="Sacherklärung">societas<note type="Sacherklärung">
<seg xml:lang="la">societas,-atis</seg>f.: Als <seg xml:lang="la">societas</seg> im römischen Rechtssystem verstand man eine Bündnisgemeinschaft, deren Gewinne und Verluste, falls nicht anders vereinbart, gleichmäßig auf alle Beteiligten Parteien aufgeteilt wurden. Bedingungen der Verteilung des Gewinns gemäß dem persönlichen Einsatz oder gemäß den Ressourcen, die ein Einzelner in die Gemeinschaft einbringt, werden vor Eintritt der <seg xml:lang="la">socii</seg> in die <seg xml:lang="la">societas</seg> beschlossen (vgl. <ref target="#Iustin.-inst.:3,25,2">Iustin. inst. 3,25,2</ref>).</note></hi>:
</l></lg>
</div></div></text></TEI>
\end{lstinputlisting}